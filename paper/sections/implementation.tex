\label{sec:implementation}

\subsection{Architecture}

The implementation consists of three main components:

\begin{enumerate}
    \item \textbf{Statistical Filter}: Core compression engine
    \item \textbf{Quality Metrics}: Objective evaluation system
    \item \textbf{Tokenizer Interface}: Pluggable token counting
\end{enumerate}

\subsection{Statistical Filter Module}

\textbf{Language}: Rust (Edition 2024, Rust 1.85+)

\textbf{File}: \texttt{rust/src/statistical\_filter.rs}

\subsubsection{Core Structure}

\begin{verbatim}
pub struct StatisticalFilter {
    config: StatisticalFilterConfig,
}

pub struct StatisticalFilterConfig {
    pub compression_ratio: f32,  // 0.3-0.7
    pub idf_weight: f64,         // 0.3
    pub position_weight: f64,    // 0.2
    pub pos_weight: f64,         // 0.2
    pub entity_weight: f64,      // 0.2
    pub entropy_weight: f64,     // 0.1
}
\end{verbatim}

\subsubsection{Key Methods}

\textbf{Compression}:
\begin{verbatim}
pub fn compress(
    &self,
    text: &str,
    tokenizer: &dyn Tokenizer
) -> String
\end{verbatim}

\textbf{Scoring}:
\begin{verbatim}
pub fn score_tokens(
    &self,
    text: &str,
    tokenizer: &dyn Tokenizer
) -> Vec<TokenImportance>
\end{verbatim}

\subsubsection{Implementation Details}

\textbf{IDF Calculation}:
\begin{verbatim}
fn calculate_idf(&self, words: &[&str]) 
    -> HashMap<&str, f64> {
    let total = words.len() as f64;
    let mut freq: HashMap<&str, usize> = HashMap::new();
    for word in words {
        *freq.entry(word).or_insert(0) += 1;
    }
    freq.into_iter()
        .map(|(w, f)| {
            let idf = (total / f as f64).ln();
            (w, idf)
        })
        .collect()
}
\end{verbatim}

\textbf{Position Scoring}:
\begin{verbatim}
fn calculate_position_importance(&self, words: &[&str]) 
    -> Vec<f64> {
    let n = words.len();
    words.iter().enumerate()
        .map(|(i, _)| {
            let pos = i as f64 / n as f64;
            if pos < 0.1 || pos > 0.9 { 1.0 }
            else if pos < 0.2 || pos > 0.8 { 0.7 }
            else { 0.5 }
        })
        .collect()
}
\end{verbatim}

\textbf{Stop Word Detection}:
\begin{verbatim}
fn is_stop_word(word: &str) -> bool {
    const STOP_WORDS: &[&str] = &[
        "the", "a", "an", "and", "or", "but",
        "in", "on", "at", "to", "for", "of",
        "with", "by", "from", "as", "is", "was",
        // ... 100+ stop words total
    ];
    STOP_WORDS.contains(&word.to_lowercase().as_str())
}
\end{verbatim}

\subsection{Quality Metrics Module}

\textbf{File}: \texttt{rust/src/quality\_metrics.rs}

\subsubsection{Structure}

\begin{verbatim}
#[derive(Debug, Clone)]
pub struct QualityMetrics {
    pub keyword_retention: f64,
    pub entity_retention: f64,
    pub vocabulary_ratio: f64,
    pub information_density: f64,
    pub overall_score: f64,
}
\end{verbatim}

\subsubsection{Calculation}

\begin{verbatim}
pub fn calculate(
    original: &str,
    compressed: &str
) -> Self {
    let orig_words = extract_words(original);
    let comp_words = extract_words(compressed);
    
    let keywords_orig = extract_keywords(&orig_words);
    let keywords_comp = extract_keywords(&comp_words);
    let keyword_retention = 
        intersection(&keywords_orig, &keywords_comp).len()
        / keywords_orig.len();
    
    let entities_orig = extract_entities(&orig_words);
    let entities_comp = extract_entities(&comp_words);
    let entity_retention = 
        intersection(&entities_orig, &entities_comp).len()
        / entities_orig.len();
    
    // ... similar for vocabulary and density
    
    let overall = 0.3 * keyword_retention
                + 0.3 * entity_retention
                + 0.2 * vocabulary_ratio
                + 0.2 * information_density;
    
    Self { /* ... */ }
}
\end{verbatim}

\subsection{Tokenizer Interface}

\textbf{File}: \texttt{rust/src/tokenizer.rs}

\subsubsection{Trait Definition}

\begin{verbatim}
pub trait Tokenizer {
    fn name(&self) -> &str;
    fn encode(&self, text: &str) -> Vec<Token>;
    fn decode(&self, tokens: &[Token]) -> String;
    fn count_tokens(&self, text: &str) -> usize;
}
\end{verbatim}

\subsubsection{Mock Implementation}

For testing and demonstration:

\begin{verbatim}
pub struct MockTokenizer;

impl Tokenizer for MockTokenizer {
    fn name(&self) -> &str {
        "mock-word-based"
    }
    
    fn count_tokens(&self, text: &str) -> usize {
        text.split_whitespace().count()
    }
    
    // ... other methods
}
\end{verbatim}

\textbf{Note}: Real tokenizers (GPT-4, Claude, Gemini) can be integrated via tokenizer-specific crates.

\subsection{Command-Line Tools}

\subsubsection{test\_statistical}

\textbf{File}: \texttt{rust/src/bin/test\_statistical.rs}

Validates compression on large datasets:

\begin{verbatim}
cargo run --release --bin test_statistical
\end{verbatim}

\textbf{Output}:
\begin{verbatim}
=== Statistical Compression Test ===

Dataset: benchmark_200_papers.txt
Original: 1,662,729 tokens (10.2 MB)
Compressed: 831,364 tokens
Savings: 831,365 tokens (50.0%)

Time: 0.92s (10.58 MB/s)

Quality Metrics:
  Overall Score: 88.6%
  Keyword Retention: 100.0%
  Entity Retention: 91.8%
  Vocabulary Diversity: 85.3%
  Information Density: 0.642

Top 10 Removed Words:
  1. "the" → 75,204 times (45.3%)
  2. "and" → 35,671 times (21.5%)
  ...
\end{verbatim}

\subsubsection{generate\_llm\_dataset}

\textbf{File}: \texttt{rust/src/bin/generate\_llm\_dataset.rs}

Creates evaluation pairs for LLM testing:

\begin{verbatim}
cargo run --release --bin generate_llm_dataset
\end{verbatim}

\textbf{Output}: 63 prompt pairs (original + compressed at 3 levels) with metadata and quality metrics, ready for GPT-4/Claude/Gemini validation.

\subsection{Performance Optimization}

\subsubsection{Memory Efficiency}

\begin{itemize}
    \item \textbf{HashMaps}: $O(u)$ space for unique words
    \item \textbf{Vec reuse}: Minimize allocations
    \item \textbf{Iterators}: Lazy evaluation where possible
\end{itemize}

\subsubsection{Time Optimization}

\begin{itemize}
    \item \textbf{Single-pass frequency}: Build frequency map in one pass
    \item \textbf{Efficient sorting}: Rust's \texttt{sort\_by} with inline comparisons
    \item \textbf{No regex}: Simple string operations only
\end{itemize}

\subsubsection{Scalability}

\begin{itemize}
    \item \textbf{Linear memory}: $O(n)$ where $n$ = word count
    \item \textbf{Log-linear time}: $O(n \log n)$ dominated by sorting
    \item \textbf{No parallel overhead}: Single-threaded is fast enough (<1s for 1.6M tokens)
\end{itemize}

\subsection{Testing Infrastructure}

\subsubsection{Unit Tests}

\begin{verbatim}
cargo test
\end{verbatim}

16 unit tests covering:
\begin{itemize}
    \item Score calculation correctness
    \item Word filtering logic
    \item Edge cases (empty, short, long)
    \item Unicode handling
    \item Configuration validation
\end{itemize}

\subsubsection{Integration Tests}

\textbf{File}: \texttt{rust/tests/integration\_tests.rs}

7 integration tests:
\begin{itemize}
    \item End-to-end compression
    \item Quality metrics validation
    \item Performance benchmarks
    \item Real data (arXiv papers)
\end{itemize}

\subsubsection{Quality Assurance}

\begin{verbatim}
cargo clippy  # 0 warnings
cargo fmt     # Auto-format
\end{verbatim}

\subsection{Deployment}

\subsubsection{Library Usage}

\begin{verbatim}
use compression_prompt::statistical_filter::{
    StatisticalFilter,
    StatisticalFilterConfig,
};
use compression_prompt::tokenizer::MockTokenizer;

let filter = StatisticalFilter::new(
    StatisticalFilterConfig::default()
);
let tokenizer = MockTokenizer;
let compressed = filter.compress(&text, &tokenizer);
\end{verbatim}

\subsubsection{Configuration Presets}

\begin{verbatim}
// Conservative (high quality)
let config = StatisticalFilterConfig {
    compression_ratio: 0.7,
    ..Default::default()
};

// Balanced (recommended)
let config = StatisticalFilterConfig::default();

// Aggressive (maximum savings)
let config = StatisticalFilterConfig {
    compression_ratio: 0.3,
    ..Default::default()
};
\end{verbatim}

\subsection{Optical Context Compression (Image Output)}

\textbf{Status}: BETA - Inspired by DeepSeek-OCR~\cite{deepseek2024ocr}

\subsubsection{Motivation}

Vision-capable LLMs (GPT-4V, Claude 3, Gemini Vision) can process images as context. By rendering compressed text as dense monospace images, we can leverage vision tokens instead of text tokens, potentially offering different cost-benefit trade-offs.

\subsubsection{Architecture}

\textbf{File}: \texttt{rust/src/image\_renderer.rs} (406 lines)

\begin{verbatim}
pub struct ImageRenderer {
    font_size: f32,          // 12.5pt default
    image_width: u32,        // 1024px
    image_height: u32,       // 1024px
    words_per_page: usize,   // ~2000 words
}
\end{verbatim}

\subsubsection{Key Features}

\begin{itemize}
    \item \textbf{Font Rendering}: DejaVu Sans Mono (embedded TrueType, 340 KB)
    \item \textbf{Auto Font Scaling}: 12.5pt default, scales down to 7pt if needed
    \item \textbf{Auto-pagination}: Splits into multiple pages (~2000 words each)
    \item \textbf{Output Formats}: PNG (lossless, ~1.4 MB/page), JPEG (Q85, ~460 KB/page)
    \item \textbf{Alpha Blending}: Smooth text rendering
    \item \textbf{Performance}: <50ms per image
\end{itemize}

\subsubsection{Implementation Details}

\begin{verbatim}
use ab_glyph::{FontRef, PxScale};
use image::{ImageBuffer, Rgba, RgbaImage};

pub fn render_to_png(&self, text: &str) 
    -> Result<Vec<Vec<u8>>> {
    let pages = self.paginate(text);
    pages.iter()
        .map(|page| self.render_page(page))
        .collect()
}

pub fn render_to_jpeg(&self, text: &str, quality: u8)
    -> Result<Vec<Vec<u8>>> {
    // Similar to PNG but with JPEG encoding
}
\end{verbatim}

\subsubsection{Benchmarks (RNN Paper - 9118 words)}

\begin{table}[h]
\centering
\caption{Image Output Performance}
\label{tab:image-output}
\begin{tabular}{lrrr}
\toprule
Metric & Value & Pages & Size/Page \\
\midrule
Text compression & 50\% & -- & -- \\
Words compressed & 9118 → 5156 & -- & -- \\
PNG output & 4.2 MB & 3 & 1.4 MB \\
JPEG output (Q85) & 1.5 MB & 3 & 460 KB \\
JPEG savings & 65.5\% & -- & vs PNG \\
Rendering time & <150 ms & 3 & <50 ms \\
\bottomrule
\end{tabular}
\end{table}

\subsubsection{Use Cases}

\begin{itemize}
    \item Dense document compression for vision models
    \item Cost optimization using vision tokens vs text tokens
    \item Research on optical context compression
    \item Large-scale document processing
\end{itemize}

\subsubsection{Limitations}

\begin{itemize}
    \item Requires vision-capable LLM
    \item Vision token costs vary by provider
    \item OCR errors possible (though text is rendered, not scanned)
    \item Pending extensive validation with vision models
\end{itemize}

\subsection{Code Quality}

\begin{itemize}
    \item \textbf{Rust Edition}: 2024 (nightly 1.85+)
    \item \textbf{Test Coverage}: >80\% on core logic (23/23 tests passing)
    \item \textbf{Documentation}: Comprehensive inline docs
    \item \textbf{Type Safety}: No \texttt{unsafe} blocks
    \item \textbf{Error Handling}: Graceful degradation
\end{itemize}
